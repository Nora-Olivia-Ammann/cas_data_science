\documentclass[a4paper, 10pt]{article}

% import packages
\usepackage[margin=0.2cm]{geometry}
\usepackage{pdflscape}
\usepackage{array}
\usepackage{makecell}
\usepackage{xcolor}
\usepackage{colortbl}
\usepackage{longtable}
\usepackage{titlesec}
\usepackage{float}
\usepackage{needspace}
\usepackage{graphicx}
\usepackage{hyperref}
\usepackage{setspace}
\usepackage{fancyhdr}
\usepackage{enumitem}
\usepackage{multicol}
\usepackage{amsmath}
\usepackage{graphicx}
\usepackage[table]{xcolor}
\usepackage{booktabs}   % for \toprule, \midrule, \bottomrule
\usepackage{adjustbox}  % for \begin{adjustbox}{max width=\textwidth}
\usepackage{etoolbox}
\usepackage{tikz}
\usepackage{caption}
\usetikzlibrary{arrows.meta, positioning, shapes.geometric,fit,calc}
\usepackage{pgfplots}
\usepgfplotslibrary{groupplots}
\pgfplotsset{compat=1.18}

% ---------- custom plot styles ----------
\pgfplotsset{
  acfstyle/.style={
    ycomb,
    mark=o,
    mark size=1.0pt, % <-- smaller circles
    mark options={solid, fill=white}
  }
}
\usepackage{pgfplots}
\pgfplotsset{compat=1.18}


\AtBeginEnvironment{tabular}{\tiny}
\definecolor{cDark}{HTML}{203F4A}
\definecolor{cTeal}{HTML}{1D8F7E}
\definecolor{cGold}{HTML}{E9C06A}
\definecolor{cOrange}{HTML}{F29B5C}
\definecolor{cRed}{HTML}{E85D47}

\renewcommand{\familydefault}{\sfdefault}

\newlength\tindent
\setlength{\tindent}{\parindent}
\setlength{\parindent}{0pt}
\renewcommand{\indent}{\hspace*{\tindent}}

% global list spacing (affects all levels)
\setlist{noitemsep, topsep=0pt, parsep=0pt, partopsep=0pt}

% now explicitly override indentation for each level you care about:
\setlist[itemize,1]{leftmargin=1.2em,label=\raisebox{0.5ex}{\scalebox{0.6}{$\bullet$}}}
\setlist[itemize,2]{leftmargin=1.4em}
\setlist[itemize,3]{leftmargin=0.5em,label=\raisebox{0.5ex}{\scalebox{0.6}{$\bullet$}}}
\setlist[itemize,4]{leftmargin=1.5em}


\setlist[enumerate,1]{leftmargin=1.2em}
\setlist[enumerate,2]{leftmargin=1.4em}
\setlist[enumerate,3]{leftmargin=0.5em}

\setlength{\columnseprule}{0.1pt}
% ------------------------------------------------------


% Configure makecell package
\renewcommand{\theadalign}{bc}
\renewcommand{\theadgape}{\Gape[4pt]}
\renewcommand{\cellgape}{\Gape[4pt]}
\renewcommand{\cellalign}{lt}

% global line spacing
\renewcommand{\baselinestretch}{1.2}

% colour definitions
\definecolor{lightergray}{gray}{0.90}

\providecommand{\tightlist}{%
\setlength{\itemsep}{0pt}
\setlength{\parskip}{0pt}}

\begin{document}
  \begin{landscape}
%\begin{multicols}{4}
\tiny


{\footnotesize{Wozu und weshalb MLOps?}}

\begin{itemize}
    \item {\textbf{Motivation und Grundlagen}}
    \begin{itemize}
        \item Wertuschöfpung durch KI, untersützt unternehmen in
        \item Verkauf
        \begin{itemize}
            \item Marketing / Kampagnen (reduktion Streuverlust)
            \item Automatisierte Dienstleisungen (Bessere Informationen)
            \item Daten-Produkte als Service (Wertschöpfung)
        \end{itemize}
        \item Betrieb
        \begin{itemize}
            \item Optimierte Prozesse (Elimination von Abfall)
            \item Effizienter Mitteleinsatzt (Prognose und Allokation)
            \item Reduzzierte Risiken (Erkennen und Aktion)
        \end{itemize}
        \item Durch Datengestützte Entscheidungen und automatisierte Abläufe
    \end{itemize}
    \item {\textbf{Realität: Die meisten KI Projekte erbringen keinen Mehrwert}} Gründe Fehlender Erfolg
    \begin{itemize}
        \item Ziel-Bingo (Wiedersprüchliche Ziele, keine oder ungeeignete kriterien
        \item Datenqualität (Fokus auf Modelle, zielorientierte Nutzung von Daten wichtiger)
        \item Erwartungen überdreht (KI is komplex, kein Wundermittel)
        \item Zuviel Technologie (wirtschaftlicher Mehrwert muss A \& O sein)
    \end{itemize}
    \item \textbf{Was und Wozu?}
    \begin{itemize}
        \item Was ist das Ziel?
        \item Welche Mittel sind erfoderlich um dieses Ziel zu erreichen?
        \item Welche Infrastruktur?
        \item Wie wird gemessen ob das Ziel erreicht wird?
    \item \textbf{Beispiel eines Produktes}
    \begin{itemize}
        \item Was ist das Ziel?
        \begin{itemize}
            \item Betriebwirschaftliches Ziel: Mehr umsatz pro Kunde
            \item Erwartete Aktion (was soll es bewirken)?: Kunde produkte empfehlen die anderen auch gefallt
            \item Erfordertliche Entscheidung (welche prediction durch KI)?: Liste top-10 Produkte
            \item Business Metrik (KPI): 15\% umsatzsteigerung A (ohne) / B (mit) test
            \item AI/ML Metrik (Qualität): \% Accuracy, mindestens 1 \%
        \end{itemize}
        \item Wie erreichen wir das Ziel?
        \begin{itemize}
            \item Prozess vs. Daten (Beobachtungen, Profile) Schlussendliche Recommender Modell
        \end{itemize}
        \item Welche Systeme sind erforderlich?
        \begin{itemize}
            \item AI System: Reports, Workflows, Analyse + ML Modell, Data Lake
            \item Operatives System / Prozess: Shop System, CRM System
        \end{itemize}
    \end{itemize}
    \end{itemize}
    \item DevOps nicht gleich MLOps weil modelle aus daten bestehen nicht code und devops ist kompliziert
    \item {\textbf{DevOps: Bereitstellung von Code}}
    \begin{itemize}
        \item Statisch: Daten + Code ändern sich langsam und werden durch Anforderungen gesteuert
        \item Produktion-Silo: Daten sind hauptsächlich für die Produktion relevant
        \item Minimale Ressourcen: Produktionsressourcen \(>\) Dev-Ressourcen
        \item "Keep-alive" Ops: Überwachung aus System ebene nicht Inhalte
        \item Gut: Code + Modell versioniert und als release verpackt
        \item Schlecht: Schwergewichtig mit Fullstack, online Lernen mit mehrfachversionen nicht einfach
    \end{itemize}
    \item {\textbf{MLOps: Bereitstellung von Daten + Modellen + Code}}
    \begin{itemize}
        \item Dynamisch: Daten erzeugen Funktion, unerwartet und plötzlich
        \item Produktions-Parität: Daten sind Schlüssel in Entwicklung + Produktion
        \item Maximale Ressourcen: Trainingsressourcen \(>\) Live-System
        \item Value Ops: Überwachung auf Daten-, Modell- und Systemebene
        \item ML-Systeme werden anhand von Geschäftszielen entwickelt und in drei Phasen iterativ operationalisiert: Development \(\rightarrow\) Delivery \(\rightarrow\) Operations \(\rightarrow\) beginning
        \item MLOps umfasst einen prozess und eine Ingrastruktur um Entwicklung, Lieferung / Installation und den Betrieb ink. Qualitätssicherung (teil-)automatisieren
        \item MLOps Focus 1:  Operationalize \& Run: Get to production \& run efficiently
        \item MLOps Focus 2:  Keep it running + Quality assurance: ensure results stay valid and improve
        \item MLOps als Prozess
        \begin{itemize}
        \item Gewünschte Eigenschaften: Zielgerichtet, messbare Ergebnisse, Wiederholbar, Versioniert
        \item Um dies zu erreichen: Definierte Schritte, Automatisierte Prozesse 8Code), Resultate sind gespeichert, Einheitliche Infrastruktur
        \item Gut: Modelle als Daten, sofortige bereitstellung und mehrere Versionen, standardisiertes Engineering
        \item Schlecht: Innovativ muss erklärt werden
        \end{itemize}
    \end{itemize}
\end{itemize}


\vspace{0.4em}
\hrule
\vspace{0.7em}
{\footnotesize{Entwicklung / Betrieb}}

\begin{itemize}
    \item {\textbf{ML-Entwicklulng}}
    \begin{itemize}
        \item Wenige personen
        \item direkter sporadischer Zugriff auf datenquelle
        \item Ad-hoc Code, Modelle, Ergebnisse
        \item Keine / wenig Nachverfolgung
        \item Test Validierung nach Bedarf
        \item Schwer Reproduzierbar
    \end{itemize}
    \item {\textbf{ML-Betrieb}}
    \begin{itemize}
        \item Automatisierte Luafzeit
        \item permanenter Datenzugriff und Rechenleistung
        \item Getestete Sofware
        \item Versionierte Skripts, Modelle, Ergebnisse
        \item Live-Überwachung
        \item Reproduzierbar, wartbar
    \end{itemize}
    \item Schritte zur Produktion: Entwicklungsumbebung (lokal) \(\rightarrow\) Gespeichertes ML-Modell als datei \(\rightarrow\) Modell Server meist mit REST API
    \item Model Server:
    \begin{itemize}
        \item Modell wird bei start geladen
        \item Client sendet anfrage
        \item REST API nimmt entgegen
        \item Features aus Daten extrahiert und für Modell aufbereitet
        \item Modell predict
        \item resultat zu client
    \end{itemize}
    \item {\textbf{Model Serving Patterns}}
    \begin{itemize}
        \item Wann wird das Modell trainiert (offline / online)
        \item Wann wird die Prognose erstellt (on-demand / batch)
    \end{itemize}
\end{itemize}


%\end{multicols}
\end{landscape}
\end{document}
