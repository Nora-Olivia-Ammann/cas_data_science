\documentclass[a4paper, 10pt]{article}

% import packages
\usepackage[margin=0.2cm]{geometry}
\usepackage{pdflscape}
\usepackage{array}
\usepackage{makecell}
\usepackage{xcolor}
\usepackage{colortbl}
\usepackage{longtable}
\usepackage{titlesec}
\usepackage{float}
\usepackage{needspace}
\usepackage{graphicx}
\usepackage{hyperref}
\usepackage{setspace}
\usepackage{fancyhdr}
\usepackage{enumitem}
\usepackage{multicol}
\usepackage{amsmath}
\usepackage{graphicx}
\usepackage[table]{xcolor}
\usepackage{booktabs}   % for \toprule, \midrule, \bottomrule
\usepackage{adjustbox}  % for \begin{adjustbox}{max width=\textwidth}
\usepackage{etoolbox}
\usepackage{tikz}
\usepackage{caption}
\usetikzlibrary{arrows.meta, positioning, shapes.geometric,fit,calc}
\usepackage{pgfplots}
\usepgfplotslibrary{groupplots}
\pgfplotsset{compat=1.18}

% ---------- custom plot styles ----------
\pgfplotsset{
  acfstyle/.style={
    ycomb,
    mark=o,
    mark size=1.0pt, % <-- smaller circles
    mark options={solid, fill=white}
  }
}
\usepackage{pgfplots}
\pgfplotsset{compat=1.18}


\AtBeginEnvironment{tabular}{\tiny}
\definecolor{cDark}{HTML}{203F4A}
\definecolor{cTeal}{HTML}{1D8F7E}
\definecolor{cGold}{HTML}{E9C06A}
\definecolor{cOrange}{HTML}{F29B5C}
\definecolor{cRed}{HTML}{E85D47}

\renewcommand{\familydefault}{\sfdefault}

\newlength\tindent
\setlength{\tindent}{\parindent}
\setlength{\parindent}{0pt}
\renewcommand{\indent}{\hspace*{\tindent}}

% global list spacing (affects all levels)
\setlist{noitemsep, topsep=0pt, parsep=0pt, partopsep=0pt}

% now explicitly override indentation for each level you care about:
\setlist[itemize,1]{leftmargin=1.2em,label=\raisebox{0.5ex}{\scalebox{0.6}{$\bullet$}}}
\setlist[itemize,2]{leftmargin=1.4em}
\setlist[itemize,3]{leftmargin=0.5em,label=\raisebox{0.5ex}{\scalebox{0.6}{$\bullet$}}}
\setlist[itemize,4]{leftmargin=1.5em}


\setlist[enumerate,1]{leftmargin=1.2em}
\setlist[enumerate,2]{leftmargin=1.4em}
\setlist[enumerate,3]{leftmargin=0.5em}

\setlength{\columnseprule}{0.1pt}
% ------------------------------------------------------


% Configure makecell package
\renewcommand{\theadalign}{bc}
\renewcommand{\theadgape}{\Gape[4pt]}
\renewcommand{\cellgape}{\Gape[4pt]}
\renewcommand{\cellalign}{lt}

% global line spacing
\renewcommand{\baselinestretch}{1.2}

% colour definitions
\definecolor{lightergray}{gray}{0.90}

\providecommand{\tightlist}{%
\setlength{\itemsep}{0pt}
\setlength{\parskip}{0pt}}

\begin{document}
  \begin{landscape}
\begin{multicols}{4}
\tiny


{\footnotesize{Wozu und weshalb MLOps?}}

\begin{itemize}
    \item {\textbf{Motivation und Grundlagen}}
    \begin{itemize}
        \item Wertuschöfpung durch KI, untersützt unternehmen in
        \item Verkauf
        \begin{itemize}
            \item Marketing / Kampagnen (reduktion Streuverlust)
            \item Automatisierte Dienstleisungen (Bessere Informationen)
            \item Daten-Produkte als Service (Wertschöpfung)
        \end{itemize}
        \item Betrieb
        \begin{itemize}
            \item Optimierte Prozesse (Elimination von Abfall)
            \item Effizienter Mitteleinsatzt (Prognose und Allokation)
            \item Reduzzierte Risiken (Erkennen und Aktion)
        \end{itemize}
        \item Durch Datengestützte Entscheidungen und automatisierte Abläufe
    \end{itemize}
    \item {\textbf{Realität: Die meisten KI Projekte erbringen keinen Mehrwert}} Gründe Fehlender Erfolg
    \begin{itemize}
        \item Ziel-Bingo (Wiedersprüchliche Ziele, keine oder ungeeignete kriterien
        \item Datenqualität (Fokus auf Modelle, zielorientierte Nutzung von Daten wichtiger)
        \item Erwartungen überdreht (KI is komplex, kein Wundermittel)
        \item Zuviel Technologie (wirtschaftlicher Mehrwert muss A \& O sein)
    \end{itemize}
    \item \textbf{Was und Wozu?}
    \begin{itemize}
        \item Was ist das Ziel?
        \item Welche Mittel sind erfoderlich um dieses Ziel zu erreichen?
        \item Welche Infrastruktur?
        \item Wie wird gemessen ob das Ziel erreicht wird?
    \item \textbf{Beispiel eines Produktes}
    \begin{itemize}
        \item Was ist das Ziel?
        \begin{itemize}
            \item Betriebwirschaftliches Ziel: Mehr umsatz pro Kunde
            \item Erwartete Aktion (was soll es bewirken)?: Kunde produkte empfehlen die anderen auch gefallt
            \item Erfordertliche Entscheidung (welche prediction durch KI)?: Liste top-10 Produkte
            \item Business Metrik (KPI): 15\% umsatzsteigerung A (ohne) / B (mit) test
            \item AI/ML Metrik (Qualität): \% Accuracy, mindestens 1 \%
        \end{itemize}
        \item Wie erreichen wir das Ziel?
        \begin{itemize}
            \item Prozess vs. Daten (Beobachtungen, Profile) Schlussendliche Recommender Modell
        \end{itemize}
        \item Welche Systeme sind erforderlich?
        \begin{itemize}
            \item AI System: Reports, Workflows, Analyse + ML Modell, Data Lake
            \item Operatives System / Prozess: Shop System, CRM System
        \end{itemize}
    \end{itemize}
    \end{itemize}
    \item DevOps nicht gleich MLOps weil modelle aus daten bestehen nicht code und devops ist kompliziert
    \item {\textbf{DevOps: Bereitstellung von Code}}
    \begin{itemize}
        \item Statisch: Daten + Code ändern sich langsam und werden durch Anforderungen gesteuert
        \item Produktion-Silo: Daten sind hauptsächlich für die Produktion relevant
        \item Minimale Ressourcen: Produktionsressourcen \(>\) Dev-Ressourcen
        \item "Keep-alive" Ops: Überwachung aus System ebene nicht Inhalte
        \item Gut: Code + Modell versioniert und als release verpackt
        \item Schlecht: Schwergewichtig mit Fullstack, online Lernen mit mehrfachversionen nicht einfach
    \end{itemize}
    \item {\textbf{MLOps: Bereitstellung von Daten + Modellen + Code}}
    \begin{itemize}
        \item Dynamisch: Daten erzeugen Funktion, unerwartet und plötzlich
        \item Produktions-Parität: Daten sind Schlüssel in Entwicklung + Produktion
        \item Maximale Ressourcen: Trainingsressourcen \(>\) Live-System
        \item Value Ops: Überwachung auf Daten-, Modell- und Systemebene
        \item ML-Systeme werden anhand von Geschäftszielen entwickelt und in drei Phasen iterativ operationalisiert: Development \(\rightarrow\) Delivery \(\rightarrow\) Operations \(\rightarrow\) beginning
        \item MLOps umfasst einen prozess und eine Ingrastruktur um Entwicklung, Lieferung / Installation und den Betrieb ink. Qualitätssicherung (teil-)automatisieren
        \item MLOps Focus 1:  Operationalize \& Run: Get to production \& run efficiently
        \item MLOps Focus 2:  Keep it running + Quality assurance: ensure results stay valid and improve
        \item MLOps als Prozess
        \begin{itemize}
        \item Gewünschte Eigenschaften: Zielgerichtet, messbare Ergebnisse, Wiederholbar, Versioniert
        \item Um dies zu erreichen: Definierte Schritte, Automatisierte Prozesse 8Code), Resultate sind gespeichert, Einheitliche Infrastruktur
        \item Gut: Modelle als Daten, sofortige bereitstellung und mehrere Versionen, standardisiertes Engineering
        \item Schlecht: Innovativ muss erklärt werden
        \end{itemize}
    \end{itemize}
\end{itemize}


\vspace{0.4em}
\hrule
\vspace{0.7em}
{\footnotesize{Entwicklung / Betrieb}}

\begin{itemize}
    \item {\textbf{ML-Entwicklulng}}
    \begin{itemize}
        \item Wenige personen
        \item direkter sporadischer Zugriff auf datenquelle
        \item Ad-hoc Code, Modelle, Ergebnisse
        \item Keine / wenig Nachverfolgung
        \item Test Validierung nach Bedarf
        \item Schwer Reproduzierbar
    \end{itemize}
    \item {\textbf{ML-Betrieb}}
    \begin{itemize}
        \item Automatisierte Luafzeit
        \item permanenter Datenzugriff und Rechenleistung
        \item Getestete Sofware
        \item Versionierte Skripts, Modelle, Ergebnisse
        \item Live-Überwachung
        \item Reproduzierbar, wartbar
    \end{itemize}
    \item Schritte zur Produktion: Entwicklungsumbebung (lokal) \(\rightarrow\) Gespeichertes ML-Modell als datei \(\rightarrow\) Modell Server meist mit REST API
    \item Model Server:
    \begin{itemize}
        \item Modell wird bei start geladen
        \item Client sendet anfrage
        \item REST API nimmt entgegen
        \item Features aus Daten extrahiert und für Modell aufbereitet
        \item Modell predict
        \item resultat zu client
    \end{itemize}
    \item {\textbf{Model Serving Patterns}}
    \begin{itemize}
        \item Wann wird das Modell trainiert (offline / online)
        \item Wann wird die Prognose erstellt (on-demand / batch)
    \end{itemize}
\end{itemize}


\vspace{0.4em}
\hrule
\vspace{0.7em}
{\footnotesize{Woraus besteht ein KI System?}} Prozess und Arhitektur

\begin{itemize}
    \item Reihe von Komponenten, die entlang eines Prozesses integriert sind um ein bestimmmtes Ziel wiederkehrend nachhaltig und soweit wie möglich automatisiert zu erreichen.
    \item System Komponente
    \begin{itemize}
        \item \textbf{Data Lake} Ingest \& prepare Data (data Egineer)
        \item \textbf{Analyse + ML Modell} Build \& Operate Model \(\rightarrow\) going to Deliver
        \item \textbf{Report + Workflow} Deliver AI System
    \end{itemize}
    \item 5 kerfragen die zu den Bausteinen Führen
    \begin{enumerate}
        \item {\textbf{Datenspeicherung und Zugang?}} Datensets, Models
        \item {\textbf{Wo ausführen?}} Training, Evaluation, Validierung, Test
        \item {\textbf{Wie ausliefern und betreiben?}} Model Serving, Applications Services, Dashnoards, Apps
        \item {\textbf{Wie nachvollziehen \& überwachen?}} Metadata, Experimente, Monitoring, Logging
        \item {\textbf{Wie skalieren?}} On-prem oder Cloud, viele Modelle, anzahl persoonen, nachvollziebares training
    \end{enumerate}
\end{itemize}


\vspace{0.4em}
\hrule
\vspace{0.7em}

{\footnotesize{Qualitätssicherung für KI Systeme}}

\begin{itemize}
    \item Offline evaluation (Entwicklung \& test)
    \begin{itemize}
        \item Modell-Korrektheit (Classifisers, Regression, Overfitting, underfitting, confidence 
        \item 3 Teilige Daten Training, Val, Test
    \end{itemize}
    \item Online Evaluation (Produktiver Betrieb)
    \begin{itemize}
        \item Betriebswirtschaftliche Metriken gemäss Zielen
        \item Modell-Drift erkennen: metriken von test erfassen und überwachen, A/B tests
    \end{itemize}
    \item Herausforderungen:
    \begin{itemize}
        \item Schwankende Qualität: qualität der Ergebnisse ohne dass das system zusammenbricht
        \item Ungeplante Änderungen: Datenänderungn, unerwartet, langsam, intermittierend oder abrupt kommen
        \item Fehler schlecht erkennbar: subtile fehler
        \item Fehler schwer beheben: kann eine neue analyse und training erfordern
    \end{itemize}
    \item Funktioniert das System? -- System Monitoring  -- uptime
    \item Sind die Daten in guter Qualität? -- Data Monitoring -- verteilung input-daten
    \item Sind Prognosen korrekt? -- Prediction Monitoring -- feedback, analyse resultate
    \item Erzeugt das System Nutzen? -- Business Monitoring  -- umsatz, kosten
    \item Drift
    \begin{itemize}
        \item Data Drift: distribution of data inputs that it was trained on is different from the distribution of the data inputs that it is applied to. Changes in user demographics on an e-commerce site
        \item Concept / Model Drift: Functional relationship change between model input and output data. Model functions the same but context changed. Seasonal changes in product demand
        \item Label Drift: distribution of target variable changes. eg A change in the proportion of spam to non-spam emails over time.
        \item Feature Drift: changes distribution of individual features: A sudden increase in a particular product’s sales due to a viral marketing campaign, affecting the sales feature
    \end{itemize}
    \item Erkennung:
    \begin{itemize}
        \item Metriken über Zeit sammeln und vergleichen
        \item Gezielte Analyse einzelner Fälle
        \item Versionierung von Daten \& Modellen (Basisversion erhalten "canary", Input/Output-Daten sammeln "decoy", A/B Testung "rendezvous architecture"
    \end{itemize}
\end{itemize}

\begin{itemize}
  \item \textbf{Herausforderung: Alles kann sich ändern}
    \begin{itemize}
      \item Zwischen Vorhersage und Bewertung kann viel Zeit vergehen (vielleicht erfahren wir nie, ob die Vorhersage korrekt war).
      \item Aktuelle Daten können anders aussehen als die historischen Daten während dem Training.
      \item Aktuelle Daten können sich über Zeit verändern, manchmal langsam, manchmal schnell (oder gar nicht) (z.\,B.\ preferences change, economic environment different).
      \item Vorhersagen können anders aussehen, als gemäss historischen Daten erwartet.
    \end{itemize}

  \item \textbf{Monitoring soll Vergleiche ermöglichen}
    \begin{itemize}
      \item Prognosen : Wahrheit $\Rightarrow$ Metrik $\Delta$ (Delta)
      \item Prognose : Training $\Rightarrow$ Metrik $\Delta$ (Genauigkeit)
      \item Historische Daten : Neue Daten $\Rightarrow$ Feature $\Delta$
      \item Aktuelle Daten : Neue Daten $\Rightarrow$ Feature $\Delta$
    \end{itemize}      
    \item $\Rightarrow$ Signifikante $\Delta$ kann Drift signalisieren
        \begin{itemize}
          \item Prediction Drift (concept drift, model drift)
          \item Feature Drift (data drift)
        \end{itemize}
\end{itemize}



\vspace{0.4em}
\hrule
\vspace{0.7em}

{\footnotesize{Produkte Services auswählen}}

\begin{itemize}
    \item Consider all aspects, not just ML that is too narrow
    \item Key Features
    \begin{itemize}
        \item Value centric: steed to market, fast iteration, automated infrastructure
        \item Ease of use: dev \& ops of data products, e2e
        \item Low engineering: integrated DataOps, MLOps, DevOps principles
        \item Speedy and scalable operations: laptop to on-prem to cloud
    \end{itemize}
\end{itemize}

\vspace{0.4em}
\hrule
\vspace{0.7em}

{\footnotesize{Organistaotirscher Ansatz Qualitätssicherung}}


{\textbf{Modell-Dokumentation}} Model Card, System Card

\begin{itemize}
  \item \textbf{Kontext}
    \begin{itemize}
      \item Ziel (Business-Sicht)
      \item Beschreibung (Funktion in Bezug auf Business \& Technisch)
      \item Verantwortlichkeiten (Datenschutz, Sicherheit, Gesetzliches)
    \end{itemize}

  \item \textbf{Daten und Prozesse}
    \begin{itemize}
      \item Verwendete Daten (Beschreibung, Transformationen, relevante Analysen \& Erkenntnisse)
      \item Trainings- und Evaluationsprozess
      \item Prozess für Drift-Erkennung
    \end{itemize}

  \item \textbf{Qualität und Nutzung}
    \begin{itemize}
      \item Validierung \& Metriken (Modell-Qualität)
      \item Ergebnis-Interpretation und Validierung (Link zu Business)
      \item Nutzungsleitlinie, Beispiele (Anwendungsfälle, Grenzen)
    \end{itemize}

  \item \textbf{Technische Unterstützung}
    \begin{itemize}
      \item Technische Beschreibung (Modell, Schnittstellen, Sicherheit)
      \item Mathematische Beschreibung, Annahmen, Limiten
      \item Referenzen, Literatur
    \end{itemize}
\end{itemize}

{\textbf{Erfolgsfaktoren}}

\begin{itemize}
  \item \textbf{Produkt/Projekte-Ebene}
    \begin{itemize}
      \item Klar definierte, messbare Ziele
      \item Daten sind verfügbar \& in guter Qualität
      \item Fachübergreifendes Team -- Business \& IT
      \item Data Science Platform (DataOps, MLOps) ist als Infrastruktur verfügbar
      \item Budget \& Zeit sind verfügbar
    \end{itemize}

  \item \textbf{Organisatorische Ebene}
    \begin{itemize}
      \item Motivation ist betrieblicher Mehrwert
      \item Wille zu lernen: Experimente und(!) Fehler
      \item Erlaubnis, daten-orientiert neue Wege zu gehen
      \item Führungsebene entwickelt Daten-Kompetenz für sich \& Unternehmung weiter
    \end{itemize}
\end{itemize}


\vspace{0.4em}
\hrule
\vspace{0.7em}

{\footnotesize{MLOps Conclusion}}

\begin{itemize}
  \item \textbf{MLOps is more than DevOps + ML:}
    \begin{itemize}
      \item Focus on end-2-end process
      \item Ensure continuous tracking \& monitoring
      \item Layered, modular architecture for efficiency
    \end{itemize}

  \item \textbf{MLOps Building Blocks}
    \begin{itemize}
      \item Feature Store (pipelines, online/offline, reproducible)
      \item Metadata (experiments, data, models)
      \item Model Serving (online, streaming, batch)
      \item Model Monitoring (concept, data drift, triggers)
    \end{itemize}

  \item \textbf{MLOps is a Tool, not Objective}
    \begin{itemize}
      \item Business value remains the key driver
      \item MLOps enables efficiency, process, governance
      \item Narrow point solutions increase integration cost
      \item Want: Integrated data product delivery capability
    \end{itemize}

  \item \textbf{Make or Buy?}
    \begin{itemize}
      \item Market very dynamic, many new vendors
      \item No established end-2-end, consolidation likely
      \item Licensing often focussed on cost of use, not value
      \item Risking vendor lock-in
    \end{itemize}

  \item \textbf{Key Insights}
    \begin{itemize}
        \item Team verantwortungen
        \begin{itemize}
            \item Data Scientist: These \(\rightarrow\) Test
            \item ML Engineer: Problem \(\rightarrow\) Lösung
            \item Product Owner: Verantwortung
        \end{itemize}
      \item Business value should drive, not technology
      \item Leverage ready-made baseline technologies
      \item Focus on open architecture, preferring open source
      \item Alternatively, leverage ready-made cloud/on-prem platforms or end-2-end solutions for your use case
      \item Carefully evaluate acceptable level of vendor lock-in
    \end{itemize}
\end{itemize}



\end{multicols}
\end{landscape}
\end{document}
